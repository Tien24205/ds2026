\documentclass[a4paper,12pt]{article}
\usepackage{graphicx}
\usepackage{float}
\usepackage{listings}
\usepackage{xcolor}

\title{Practical Work 2: RPC File Transfer}
\author{USTH -- Distributed Systems \\ Group Members: <your names>}
\date{\today}

\lstset{
    basicstyle=\ttfamily\small,
    frame=single,
    breaklines=true
}

\begin{document}
\maketitle

\section{Introduction}
In Practical Work 1, we developed a TCP-based file transfer system using sockets.  
In Practical Work 2, we upgrade the system to use RPC (Remote Procedure Calls).  
RPC allows the client to call a remote function on the server without handling low-level sockets.

We implemented the system using \textbf{XML-RPC}, a simple built-in RPC library in Python.  
It requires no additional installation and is easy to understand.

\section{RPC Service Design}

The RPC service exposes one remote method:

\begin{center}
\texttt{get\_file(filename) → returns file chunks}
\end{center}

The logic is simple:

\begin{itemize}
    \item Client calls the RPC method requesting a filename.
    \item Server checks if the file exists.
    \item If the file does not exist, the server returns an error.
    \item If the file exists, the server reads it in 4096-byte chunks.
    \item Each chunk is Base64 encoded (XML-RPC does not support raw binary).
    \item The client decodes the chunks and reconstructs the file.
\end{itemize}

\subsection*{RPC Sequence Diagram}
\begin{figure}[H]
    \centering
    \includegraphics[width=0.95\textwidth]{rpc_sequence.png}
    \caption{RPC File Transfer Sequence Diagram}
\end{figure}

\section{System Organization}

The system is organized into two main components:

\begin{itemize}
    \item \textbf{RPC Server:} Hosts the \texttt{get\_file()} function and returns file data.
    \item \textbf{RPC Client:} Calls the RPC function and writes the received file to disk.
\end{itemize}

\begin{figure}[H]
    \centering
    \includegraphics[width=0.95\textwidth]{rpc_architecture.png}
    \caption{System Architecture Using RPC}
\end{figure}

Compared to TCP sockets, RPC simplifies:

\begin{itemize}
    \item Connection handling
    \item Message formatting
    \item Error management
    \item File chunk transmission
\end{itemize}

\section{Implementation}

\subsection{RPC Server Code (Python)}
\begin{lstlisting}[language=Python]
from xmlrpc.server import SimpleXMLRPCServer
import base64
import os

CHUNK = 4096

def get_file(filename):
    if not os.path.exists(filename):
        return {"error": "File not found"}

    result = []
    with open(filename, "rb") as f:
        while True:
            data = f.read(CHUNK)
            if not data:
                break
            result.append(base64.b64encode(data).decode("utf-8"))
    return {"chunks": result}

server = SimpleXMLRPCServer(("0.0.0.0", 5000))
print("RPC Server running on port 5000...")
server.register_function(get_file, "get_file")
server.serve_forever()
\end{lstlisting}

\subsection{RPC Client Code (Python)}
\begin{lstlisting}[language=Python]
import xmlrpc.client
import base64

proxy = xmlrpc.client.ServerProxy("http://127.0.0.1:5000/")

filename = input("Enter filename to download: ")

response = proxy.get_file(filename)

if "error" in response:
    print("Server:", response["error"])
else:
    with open("received_" + filename, "wb") as f:
        for chunk in response["chunks"]:
            f.write(base64.b64decode(chunk))
    print("File downloaded successfully!")
\end{lstlisting}

\section{Connection to TCP Version}
Below is a simplified excerpt of the original TCP code used in Practical Work 1.

\subsection*{Original TCP Server (excerpt)}
\begin{lstlisting}[language=C]
while ((bytes = fread(buffer, 1, BUFFER_SIZE, file)) > 0) {
    send(client_socket, buffer, bytes, 0);
}
\end{lstlisting}

\subsection*{Original TCP Client (excerpt)}
\begin{lstlisting}[language=C]
while (total_received < file_size) {
    bytes = recv(server, buffer, BUFFER_SIZE, 0);
    fwrite(buffer, 1, bytes, file);
}
\end{lstlisting}

These manual socket operations are replaced by an RPC function call.

\section{Work Distribution}
\begin{itemize}
    \item \textbf{Student A:} Implemented RPC server.
    \item \textbf{Student B:} Implemented RPC client.
    \item \textbf{Student C:} Created diagrams and protocol design.
    \item \textbf{Student D:} Wrote the LaTeX report.
\end{itemize}

\section{Conclusion}
The RPC version of our file transfer system is significantly simpler than the TCP version.  
RPC hides socket management and lets us focus on the functionality: transferring a file.  
Using XML-RPC made the implementation lightweight and easy to understand.

\end{document}
