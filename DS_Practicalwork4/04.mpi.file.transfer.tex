\documentclass[a4paper,12pt]{article}
\usepackage[T1]{fontenc}
\usepackage[utf8]{inputenc}
\usepackage{graphicx}
\usepackage{float}
\usepackage{listings}
\usepackage{xcolor}

\title{Practical Work 4: Word Count using MapReduce}
\author{USTH -- Distributed Systems \\ Do Minh Tien - 23BI14421}
\date{\today}

\lstset{
    basicstyle=\ttfamily\small,
    frame=single,
    breaklines=true
}

\begin{document}
\maketitle

\section{Introduction}
This practical work implements the classical Word Count program using a
custom MapReduce model in Python.  
The goal is to understand how Mapper, Shuffle, and Reducer components
cooperate in distributed data processing, without relying on heavy
frameworks such as Hadoop or Spark.

The assignment requires using a MapReduce system of our choice,
and since Python is allowed while C/C++ frameworks are unavailable,
we developed our own lightweight MapReduce pipeline.

\section{Why We Chose Our MapReduce Implementation}
We implemented a minimal MapReduce system in Python because:

\begin{itemize}
    \item Python is simple, portable, and installed on all USTH lab machines.
    \item It avoids the heavy installation and configuration of Hadoop.
    \item It clearly demonstrates the internal logic of Map, Shuffle, and Reduce.
    \item The pipeline can be executed using Unix tools (cat, sort).
    \item It satisfies the requirement of “inventing” our own MapReduce version.
\end{itemize}

This makes the implementation educational, lightweight, and easy to test.

\section{MapReduce Architecture for Word Count}

The system follows the classical 3-phase MapReduce pipeline:

\begin{enumerate}
    \item \textbf{Map Phase}:  
    Tokenizes input text and outputs \texttt{(word, 1)} pairs.

    \item \textbf{Shuffle/Sort Phase}:  
    Groups pairs by word and sorts them.

    \item \textbf{Reduce Phase}:  
    Sums all counts for each word and emits final totals.
\end{enumerate}

\begin{figure}[H]
    \centering
\begin{figure}
        \centering
        \includegraphics[width=0.5\linewidth]{figure.png}
        \caption{Enter Caption}
        \label{fig:placeholder}
    \end{figure}
        \caption{Word Count MapReduce Workflow}
\end{figure}

\section{Mapper Implementation}

The Mapper reads input from standard input, splits into words,
and emits a key-value pair for each occurrence:

\begin{lstlisting}[language=Python]
# mapper.py
import sys

for line in sys.stdin:
    line = line.strip().lower()
    for word in line.split():
        print(f"{word}\t1")
\end{lstlisting}

\section{Reducer Implementation}

The Reducer receives sorted pairs and aggregates word counts:

\begin{lstlisting}[language=Python]
# reducer.py
import sys

current_word = None
current_count = 0

for line in sys.stdin:
    word, count = line.strip().split("\t")
    count = int(count)

    if word == current_word:
        current_count += count
    else:
        if current_word is not None:
            print(f"{current_word}\t{current_count}")
        current_word = word
        current_count = count

if current_word is not None:
    print(f"{current_word}\t{current_count}")
\end{lstlisting}

\section{How to Run the Program}

The entire Word Count pipeline is executed using a simple Unix command:

\begin{verbatim}
cat input.txt | python3 mapper.py | sort | python3 reducer.py
\end{verbatim}

This simulates a full MapReduce workflow with:
\begin{itemize}
    \item Python scripts as Mapper and Reducer.
    \item Unix \texttt{sort} as the Shuffle/Sort step.
\end{itemize}

\section{Who Does What}

\begin{itemize}
  \item \textbf{Mapper}: Processes input chunks, tokenizes into words,
        and outputs key-value pairs (word, 1) for each occurrence.

  \item \textbf{Shuffle/Sort}: Groups all pairs by key (word)
        and sorts the keys, preparing for reduction
        (handled by \texttt{defaultdict} and \texttt{sorted}).

  \item \textbf{Reducer}: Aggregates values for each key
        by summing the list of 1s, producing the final count.

  \item \textbf{Main Function}: Splits input into chunks,
        orchestrates map/shuffle/reduce phases,
        and collects results into a dictionary.
\end{itemize}

\section{Conclusion}

This practical work demonstrates how distributed data processing can be
simulated using a lightweight MapReduce model.  
By implementing the Mapper, Shuffle, and Reducer manually in Python,
we gain a clear understanding of how large-scale systems like Hadoop
operate internally.

\end{document}
\documentclass[a4paper,12pt]{article}
\usepackage[T1]{fontenc}
\usepackage[utf8]{inputenc}
\usepackage{graphicx}
\usepackage{float}
\usepackage{listings}
\usepackage{xcolor}

\title{Practical Work 4: Word Count using MapReduce}
\author{USTH -- Distributed Systems \\ Do Minh Tien - 23BI14421}
\date{\today}

\lstset{
    basicstyle=\ttfamily\small,
    frame=single,
    breaklines=true
}

\begin{document}
\maketitle

\section{Introduction}
This practical work implements the classical Word Count program using a
custom MapReduce model in Python.  
The goal is to understand how Mapper, Shuffle, and Reducer components
cooperate in distributed data processing, without relying on heavy
frameworks such as Hadoop or Spark.

The assignment requires using a MapReduce system of our choice,
and since Python is allowed while C/C++ frameworks are unavailable,
we developed our own lightweight MapReduce pipeline.

\section{Why We Chose Our MapReduce Implementation}
We implemented a minimal MapReduce system in Python because:

\begin{itemize}
    \item Python is simple, portable, and installed on all USTH lab machines.
    \item It avoids the heavy installation and configuration of Hadoop.
    \item It clearly demonstrates the internal logic of Map, Shuffle, and Reduce.
    \item The pipeline can be executed using Unix tools (cat, sort).
    \item It satisfies the requirement of “inventing” our own MapReduce version.
\end{itemize}

This makes the implementation educational, lightweight, and easy to test.

\section{MapReduce Architecture for Word Count}

The system follows the classical 3-phase MapReduce pipeline:

\begin{enumerate}
    \item \textbf{Map Phase}:  
    Tokenizes input text and outputs \texttt{(word, 1)} pairs.

    \item \textbf{Shuffle/Sort Phase}:  
    Groups pairs by word and sorts them.

    \item \textbf{Reduce Phase}:  
    Sums all counts for each word and emits final totals.
\end{enumerate}

\begin{figure}[H]
    \centering
    \includegraphics[width=0.92\textwidth]{wordcount_architecture.png}
    \caption{Word Count MapReduce Workflow}
\end{figure}

\section{Mapper Implementation}

The Mapper reads input from standard input, splits into words,
and emits a key-value pair for each occurrence:

\begin{lstlisting}[language=Python]
# mapper.py
import sys

for line in sys.stdin:
    line = line.strip().lower()
    for word in line.split():
        print(f"{word}\t1")
\end{lstlisting}

\section{Reducer Implementation}

The Reducer receives sorted pairs and aggregates word counts:

\begin{lstlisting}[language=Python]
# reducer.py
import sys

current_word = None
current_count = 0

for line in sys.stdin:
    word, count = line.strip().split("\t")
    count = int(count)

    if word == current_word:
        current_count += count
    else:
        if current_word is not None:
            print(f"{current_word}\t{current_count}")
        current_word = word
        current_count = count

if current_word is not None:
    print(f"{current_word}\t{current_count}")
\end{lstlisting}

\section{How to Run the Program}

The entire Word Count pipeline is executed using a simple Unix command:

\begin{verbatim}
cat input.txt | python3 mapper.py | sort | python3 reducer.py
\end{verbatim}

This simulates a full MapReduce workflow with:
\begin{itemize}
    \item Python scripts as Mapper and Reducer.
    \item Unix \texttt{sort} as the Shuffle/Sort step.
\end{itemize}

\section{Who Does What}

\begin{itemize}
  \item \textbf{Mapper}: Processes input chunks, tokenizes into words,
        and outputs key-value pairs (word, 1) for each occurrence.

  \item \textbf{Shuffle/Sort}: Groups all pairs by key (word)
        and sorts the keys, preparing for reduction
        (handled by \texttt{defaultdict} and \texttt{sorted}).

  \item \textbf{Reducer}: Aggregates values for each key
        by summing the list of 1s, producing the final count.

  \item \textbf{Main Function}: Splits input into chunks,
        orchestrates map/shuffle/reduce phases,
        and collects results into a dictionary.
\end{itemize}

\section{Conclusion}

This practical work demonstrates how distributed data processing can be
simulated using a lightweight MapReduce model.  
By implementing the Mapper, Shuffle, and Reducer manually in Python,
we gain a clear understanding of how large-scale systems like Hadoop
operate internally.

\end{document}
